\documentclass[conference]{IEEEtran}
\IEEEoverridecommandlockouts
% The preceding line is only needed to identify funding in the first footnote. If that is unneeded, please comment it out.
\usepackage{cite}
\usepackage{amsmath,amssymb,amsfonts}
\usepackage{algorithmic}
\usepackage{graphicx}
\usepackage{textcomp}
\usepackage{xcolor}
\def\BibTeX{{\rm B\kern-.05em{\sc i\kern-.025em b}\kern-.08em
    T\kern-.1667em\lower.7ex\hbox{E}\kern-.125emX}}
    
% my packages

\usepackage[ruled,vlined,linesnumbered]{algorithm2e}
\usepackage{tikz}
\usepackage{pgfplots}
\usepgfplotslibrary{groupplots}
\pgfplotsset{compat=1.17} 
% \usepackage{hyperref}
% \usepackage{cleveref}
\usepackage{caption}
\usepackage{subcaption}
% \usepackage[caption=false]{subcaption}
% \captionsetup{style=base}
% \usepackage[caption=false, ...]{subfig}
% \usepackage[caption=false]{subfig}


% debug
\pagestyle{plain}

% 1. label of graph
% 2. picture
% 3. ref
% 4. perf on spiedie
% 5. host supported sketch

% the batches sync

\title{gpupaper}
\author{kchiu }
\date{May 2021}

\begin{document}
\begin{figure}
     \centering
     \begin{subfigure}[b]{0.48\columnwidth}
         \centering
        %  \includegraphics[width=\textwidth]{graph1}
        \begin{tikzpicture}[scale=0.48]
            \label{fig:thread_perf_m_0.5}
            \begin{semilogxaxis}[
                title={Insert Performance for Different m},
                xlabel={Work Load},
                ylabel={Insert Performance (M operations/s)},
                legend pos=north east,
                xmajorgrids=true,
                ymajorgrids=true,
            ]
                \addplot table[x = WorkLoad, y = m1_insert] {data/qsketch/perf_m_0.5_r.dat};
                \addlegendentry{m == 1}
                \addplot table[x = WorkLoad, y = m2_insert] {data/qsketch/perf_m_0.5_r.dat};
                \addlegendentry{m == 2}
                \addplot table[x = WorkLoad, y = m3_insert] {data/qsketch/perf_m_0.5_r.dat};
                \addlegendentry{m == 3}
            \end{semilogxaxis}
        \end{tikzpicture}
         \caption{$y=x$}
         \label{fig:y equals x}
     \end{subfigure}
     \hfill
     \begin{subfigure}[b]{0.48\columnwidth}
         \centering
        %  \includegraphics[width=\textwidth]{graph2}
        \begin{tikzpicture}[scale=0.48]
            \begin{semilogxaxis}[
                title={Search Performance for Different m},
                xlabel={Work Load},
                ylabel={Search Performance (M operations/s)},
                legend pos=north east,
                xmajorgrids=true,
                ymajorgrids=true,
            ]
                \addplot table[x = WorkLoad, y = m1_search] {data/qsketch/perf_m_0.5_r.dat};
                \addlegendentry{m == 1}
                \addplot table[x = WorkLoad, y = m2_search] {data/qsketch/perf_m_0.5_r.dat};
                \addlegendentry{m == 2}
                \addplot table[x = WorkLoad, y = m3_search] {data/qsketch/perf_m_0.5_r.dat};
                \addlegendentry{m == 3}
            \end{semilogxaxis}
        \end{tikzpicture}
         \caption{$y=3sinx$}
         \label{fig:three sin x}
     \end{subfigure}
        \caption{Three simple graphs}
        \label{fig:graphs}
\end{figure}

Figure \ref{fig:graphs} shows that
\end{document}

% \begin{figure}
%     \centering
%     \begin{subfigure}[b]{0.48\columnwidth}
    
%     \caption{}
%     \label{}
%     \end{subfigure}
%     \hfill
%     \begin{subfigure}[b]{0.48\columnwidth}
    
%     \caption{}
%     \label{}
%     \end{subfigure}
%     \caption{Caption}
%     \label{fig:my_label}
% \end{figure}