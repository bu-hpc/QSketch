\documentclass{article}
\usepackage[utf8]{inputenc}
\usepackage{blindtext}
\usepackage{multicol}
% \usepackage{listings}
% \usepackage{algorithm}
% \usepackage{algorithmic}
\usepackage[ruled,vlined]{algorithm2e}

\title{gpupaper}
\author{kchiu }
\date{May 2021}

\begin{document}

\maketitle
\begin{multicols}{2}
[
\section{First Section}
All human things are subject to decay. And when fate summons, Monarchs must obey.
]


% \blindtext\blindtext



\end{multicols}
% \begin{}{2}
\section{Introduction}
\section{Background}
\section{Related Work}
% \end{multicols}

% \begin{algorithm}
% \DontPrintSemicolon
% \KwData{$G=(X,U)$ such that $G^{tc}$ is an order.}
% \KwResult{$G’=(X,V)$ with $V\subseteq U$ such that $G’^{tc}$ is an
% interval order.}
% \Begin{
% $V \longleftarrow U$\;
% $S \longleftarrow \emptyset$\;
% \For{$x\in X$}{
% $NbSuccInS(x) \longleftarrow 0$\;
% $NbPredInMin(x) \longleftarrow 0$\;
% $NbPredNotInMin(x) \longleftarrow |ImPred(x)|$\;
% }
% \For{$x \in X$}{
% \If{$NbPredInMin(x) = 0$ {\bf and} $NbPredNotInMin(x) = 0$}{
% $AppendToMin(x)$}
% }
% \nl\While{$S \neq \emptyset$}{\label{InRes1}
% \nlset{REM} remove $x$ from the list of $T$ of maximal index\;\label{InResR}
% \lnl{InRes2}\While{$|S \cap ImSucc(x)| \neq |S|$}{
% \For{$ y \in S-ImSucc(x)$}{
% \{ remove from $V$ all the arcs $zy$ : \}\;
% \For{$z \in ImPred(y) \cap Min$}{
% remove the arc $zy$ from $V$\;
% $NbSuccInS(z) \longleftarrow NbSuccInS(z) - 1$\;
% move $z$ in $T$ to the list preceding its present list\;
% \{i.e. If $z \in T[k]$, move $z$ from $T[k]$ to
% $T[k-1]$\}\;
% }
% $NbPredInMin(y) \longleftarrow 0$\;
% $NbPredNotInMin(y) \longleftarrow 0$\;
% $S \longleftarrow S - \{y\}$\;
% $AppendToMin(y)$\;
% }
% }
% $RemoveFromMin(x)$\;
% }
% }
% \caption{IntervalRestriction\label{IR}}
% \end{algorithm}

\begin{algorithm}[H]
\DontPrintSemicolon
% \newcommand{\forcond}{$i=0$ \KwTo $n$}
% \SetKwFunction{FRecurs}{FnRecursive}%
\SetKwProg{Fn}{Function}{}{end}
\SetKwFunction{FRecurs}{FnRecursive}%

\Fn
(\tcc*[h]{algorithm as a recursive function})
{\FRecurs{some args}}
{
\KwData{Some input data\\these inputs can be displayed on several lines and one
input can be wider than line’s width.}
\KwResult{Same for output data}
\tcc{this is a comment to tell you that we will now really start code}
\If(\tcc*[h]{a simple if but with a comment on the same line}){this is true}{
we do that, else nothing\;
\tcc{we will include other if so you can see this is possible}
\eIf{we agree that}{
we do that\;
}{
else we will do a more complicated if using else if\;
\uIf{this first condition is true}{
we do that\;
}
\uElseIf{this other condition is true}{
this is done\tcc*[r]{else if}
}
\Else{
in other case, we do this\tcc*[r]{else}
}
}
}
% \tcc{now loops}
% \For{\forcond}{
% a for loop\;
% }
\While{$i<n$}{
a while loop including a repeat--until loop\;
\Repeat{this end condition}{
do this things\;
}
}
They are many other possibilities and customization possible that you have to
discover by reading the documentation.
}
\end{algorithm}

\end{document}
